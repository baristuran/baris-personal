% Created 2025-06-17 Tue 20:13
% Intended LaTeX compiler: pdflatex
\documentclass[beamer, aspectratio=1610]{beamer}
\usepackage[utf8]{inputenc}
\usepackage[T1]{fontenc}
\usepackage{graphicx}
\usepackage{longtable}
\usepackage{wrapfig}
\usepackage{rotating}
\usepackage[normalem]{ulem}
\usepackage{amsmath}
\usepackage{amssymb}
\usepackage{capt-of}
\usepackage{hyperref}
\usepackage{color}
\usepackage{animate}
\newcommand{\cvtime}[1]{\hfill \textsf{(#1)}}
\usepackage{tikz}
\usepackage{tikzsymbols}
\usetikzlibrary{decorations.pathmorphing}
\tikzset{zigzag/.style={decorate,decoration=zigzag}}
\usepackage{mathtools}
\usepackage{bm}
\usepackage{adjustbox}
\adjustboxset{width=0.99\textwidth,center,keepaspectratio}
\setbeamerfont{footnote}{size=\scriptsize}
\newcommand{\trimeqspace}{\vspace{-1.5em}}
\newcommand{\trimeqspacehalf}{\vspace{-0.75em}}
\usepackage{appendixnumberbeamer}
\usepackage{copyrightbox}
\makeatletter
\renewcommand{\CRB@setcopyrightfont}{\tiny\color{gray}}
\makeatother
\AtBeginDocument{%
\setlength{\abovedisplayskip}{6pt}%
\setlength{\belowdisplayskip}{6pt}%
\setlength{\abovedisplayshortskip}{6pt}%
\setlength{\belowdisplayshortskip}{6pt}%
}
\usepackage[backend=biber, style=verbose, giveninits=true, maxnames=99]{biblatex}
\bibliography{ref.bib}
\DeclareDelimFormat{nametitledelim}{\addspace}
\DeclareFieldFormat[article]{pages}{#1}
\DeclareBibliographyDriver{article}{%
\usebibmacro{bibindex}%
\usebibmacro{begentry}%
\usebibmacro{author/translator+others}%
\setunit{\addspace}\newblock
\printtext[parens]{\printfield{year}}%
\newunit\newblock
\usebibmacro{title}%
\newunit
\printfield{journaltitle}%
\setunit*{\addcomma\space}%
\printfield{volume}%
\setunit*{\addcomma\space}%
\printfield{pages}%
\newunit\newblock
\usebibmacro{finentry}}
\usepackage{fontawesome}
\renewcommand{\thefootnote}{\faBook}
\renewcommand{\footnotesize}{\scriptsize}
\addtobeamertemplate{footnote}{\hskip -1.5em}{}
\setbeamertemplate{blocks}[rounded][shadow]
\setbeamercolor{block title}{fg=white,bg=uniSlightblue!95}
\setbeamercolor{block body}{fg=black,bg=uniSlightblue!10}
\setbeamercolor{block title example}{fg=white,bg=uniSgreen!65!black}
\setbeamercolor{block body example}{fg=black,bg=uniSgreen!10}
\setbeamercolor{block title  alerted}{fg=white,bg=uniSred}
\setbeamercolor{block body alerted}{fg=black,bg=uniSred!10}
\setbeamercolor{title}{fg=uniSblue,bg=uniSgray!20}
\setbeamercolor{frametitle}{fg=uniSblue}
\setbeamercolor{palette primary}{bg=uniSgray!20,fg=uniSgray}
\setbeamercolor{palette secondary}{bg=uniSblue,fg=white}
\setbeamercolor{palette tertiary}{fg=white}
\setbeamercolor{palette quaternary}{fg=uniSblue}
\setbeamercolor{titlelike}{parent=palette quaternary}
\newcommand{\reynoldstress}{\boldsymbol{\tau}}
\newcommand{\reynoldstresscomp}{\mathsf{\tau}}
\newcommand{\reynoldstressa}{\boldsymbol{a}}
\newcommand{\reynoldstressacomp}{\mathsf{a}}
\newcommand{\reynoldstressb}{\boldsymbol{b}}
\newcommand{\reynoldstressbcomp}{\mathsf{b}}
\newcommand{\Tt}{\mathbf{T}}
\newcommand{\St}{\mathbf{S}}
\newcommand{\Rt}{\mathbf{R}}
\newcommand{\Omegat}{\mathbf{\Omega}}
\newcommand{\It}{\mathbf{I}}
\newcommand{\Pt}{\boldsymbol{P}}
\newcommand{\rstt}{\bm{\tau}}
\newcommand{\bu}{\mathbf{u}}
\usetheme[left]{Marburg}
\author{Hao Teng, Baris Turan, Fabian Steinbrenner}
\date{\today}
\title{Data-driven Turbulence Prediction with Latent Diffusion Models}
\title[Data-driven Turbulence Prediction with Latent Diffusion Models]{Data-driven Turbulence Prediction with Latent Diffusion Models}
\newcommand\tikzmark[1]{\tikz[remember picture,overlay] \node[inner xsep=0pt] (#1) {}; }
\author[Teng, Turan \\ Steinbrenner, Xiao  \\  ITLR-DDSim]{Hao Teng, Baris Turan, Fabian Steinbrenner, Heng Xiao}
\institute[Univeristy of Stuttgart]{Institute of Aerospace Thermodynamics \\ Faculty of Aerospace Engineering and Geodesy\\ University of Stuttgart \vspace{0.25em}}
\titlegraphic{\includegraphics[height=.15\paperheight,keepaspectratio]{logos/simtech-itlr-unistr}}
\newenvironment<>{lowkey}[1]{%
\begin{actionenv}#2%
\def\insertblocktitle{#1}%
\par%
\setbeamercolor{block title}{fg=uniSblue,bg=HStone!15}
\setbeamercolor{block body}{bg=HStone!5}
\usebeamertemplate{block begin}}{\par\usebeamertemplate{block end}\end{actionenv}}
\newenvironment<>{plain}[1]{%
\begin{actionenv}#2%
\def\insertblocktitle{#1}%
\par%
\setbeamercolor{block title}{fg=uniSblue,bg=white}
\setbeamercolor{block body}{bg=white}
\setbeamertemplate{blocks}[rounded][shadow=false]
\usebeamertemplate{block begin}}{\par\usebeamertemplate{block end}\end{actionenv}}
\usecolortheme{beaver}
\usefonttheme{professionalfonts}
\setbeamertemplate{navigation symbols}{\insertframenumber/\inserttotalframenumber}
\setbeamertemplate{caption}[numbered]
\setbeamertemplate{section in toc}[sections numbered]
\setbeamertemplate{subsection in toc}[subsections numbered]
\setbeamertemplate{caption}{\raggedright\insertcaption\par}
\definecolor{CMaroon}{RGB}{139,31,65}
\definecolor{VSunset}{RGB}{247,144,30}
\definecolor{HStone}{RGB}{109,106,117}
\definecolor{uniSblue}{RGB}{0,65,145}
\definecolor{uniSlightblue}{RGB}{0,190,255}
\definecolor{uniSgray}{RGB}{62, 68, 76}
\definecolor{uniSyellow}{RGB}{255, 213, 0}
\definecolor{uniSred}{RGB}{230, 0, 50}
\definecolor{uniSgreen}{RGB}{0, 200, 50}
\makeatletter
\setbeamertemplate{sidebar canvas \beamer@sidebarside}[vertical shading][top=uniSlightblue!90,bottom=uniSblue!75]
\makeatother
\setbeamercolor{palette sidebar secondary}{fg=black}
\setbeamercolor{section in sidebar shaded}{fg=uniSgray!85}
\setbeamercolor{subsection in sidebar shaded}{fg=uniSgray}
\setbeamercolor{subsection in sidebar}{fg=uniSblue}
\setbeamerfont{section in sidebar}{series=\bfseries}
\setbeamerfont{subsection in sidebar shaded}{series=\bfseries}
\makeatletter
\newcommand{\setnextsection}[1]{%
\setcounter{section}{\numexpr#1-1\relax}%
\beamer@tocsectionnumber=\numexpr#1-1\relax\space}
\makeatother
\hypersetup{
 pdfauthor={Hao Teng, Baris Turan, Fabian Steinbrenner},
 pdftitle={Data-driven Turbulence Prediction with Latent Diffusion Models},
 pdfkeywords={},
 pdfsubject={},
 pdfcreator={Emacs 29.3 (Org mode 9.6.15)}, 
 pdflang={English}}
\begin{document}

\maketitle
\begin{frame}{Outline}
\setcounter{tocdepth}{1}
\tableofcontents
\end{frame}


\section{Configuration\hfill{}\textsc{ignore}}
\label{sec:org119040f}
\begin{frame}[label={sec:org0fd9f34}]{Basic}

\end{frame}


\begin{frame}[label={sec:org0f069fe}]{Packages \& Setup}




\begin{block}{Customized Citation Format}

\end{block}
\end{frame}



\begin{frame}[label={sec:org008a9ca}]{VT Theme Setup}

\begin{block}{Shades for blocks in VT Theme}

\end{block}
\end{frame}


\begin{frame}[label={sec:orgb9a848a}]{Macros \& Utilities}

\end{frame}








\section{Motivation}
\label{sec:orgc6f8a8f}

\begin{frame}[label={sec:org3ea3310}]{Surrogate Modelling}
\begin{itemize}
\item 
\end{itemize}
\end{frame}

\begin{frame}[label={sec:orge7a6976}]{Limitations of Traditional Models I}
\begin{itemize}
\item 
\end{itemize}
\end{frame}

\begin{frame}[label={sec:orgf13089e}]{Limitations of Traditional Models II}
\begin{itemize}
\item 
\end{itemize}
\end{frame}


\section{A Short Introduction to Diffusion Models}
\label{sec:org2280cb6}

\begin{frame}[label={sec:orgb975cac}]{Diffusion Models: Idea}
\begin{itemize}
\item \alert{Forward process (Diffusion process)}: add noise
\begin{itemize}
\item Add i.i.d. Gaussian noise to data samples over many steps
\item Progressively degrade the data into pure noise (prior distribution, often i.i.d. Gaussian)
\item Explicitly prescribed, no modeling needed
\end{itemize}

\item \alert{Backward/Reverse process (Denoising process)}: remove noise
\begin{itemize}
\item Added noise approximated by a neural network
\item Recover the original data by gradually removing the noise using neural network, starting from pure noise
\end{itemize}
\end{itemize}
\end{frame}

\begin{frame}[label={sec:orgb356b31}]{Diffusion Models: Overall Architecture}
\begin{itemize}
\item Forward Process:
\begin{itemize}
\item Given a data sample \(x_0\), each forward step adds a small amount of Gaussian noise:
\end{itemize}
\[
  x_t = \sqrt{1 - \beta_t} x_{t-1} + \sqrt{\beta_t} \epsilon,\quad \epsilon \sim \mathcal{N}(0, I),\ t \in [1, T\bm{x}_t = \sqrt{1 - \beta_t} \bm{x}_{t-1} + \sqrt{\beta_t}
   \boldsymbol{\epsilon}, \quad \boldsymbol{\epsilon} \sim \mathcal{N}(0, \bm{I}), \ t \in [1,T]
 \]
\begin{itemize}
\item \(\beta_t\): noise schedule, \(\epsilon\): sampled Gaussian noise
\end{itemize}
\item Model Training:
\begin{itemize}
\item Get \(x_t\) from \(x_0\).
\item Train a model \(\epsilon_\theta\) to predict the added noise at a specified time step t.
\item Approximate backwards process using
\end{itemize}
\[
   x_{t-1} = \frac{1}{\sqrt{\alpha_t}} \left( x_t - \frac{1 - \alpha_t}{\sqrt{1 - \bar{\alpha}_t}} \epsilon_\theta(x_t, t) \right) + \sqrt{\beta_t} \epsilon
  \]
where \(\alpha_t = 1 - \beta_t,\quad \bar{\alpha}_t = \prod_{i=1}^t \alpha_i\)
\end{itemize}
\end{frame}


\begin{frame}[label={sec:orgab686e2}]{Diffusion Models: Conditional Diffusion Models}
\begin{itemize}
\item \emph{Diffusion model}: learn the probabilistic distribution \(p(\bm{x})\)
\item \emph{Conditional diffusion model}: learn the conditional probability of \(p(\bm{x}| \bm{y})\)
\item What \(\bm{y}\) (conditions) can be:
\begin{itemize}
\item Class labels, text prompt, attributes, or any conditioning signal \ldots{}
\end{itemize}
\item When you ask ChatGPT questions
\begin{itemize}
\item You input text \(\bm{y}\).
\item ChatGPT give you answers by sampling  \(p(\bm{x}| \bm{y})\), either text or images.
\end{itemize}
\end{itemize}
\end{frame}

\begin{frame}[label={sec:orgd9de894}]{Conditional Diffusion Models for Flow Field Prediction}
\begin{itemize}
\item In fluid dynamics, \(\bm{y}\) (conditions) can be:
\begin{itemize}
\item Labels: Reynolds number \ldots{}
\item Constraints: physical laws
\item Incomplete information: sparse measurements
\item Integral information without details: lift, drag
\end{itemize}
\item Matches the distribution of the training data, unlike deterministic models.
\end{itemize}
\end{frame}


\begin{frame}[label={sec:orgea1f652}]{Conditional Diffusion Models for Data Assimilation}
\begin{itemize}
\item 
\end{itemize}
\end{frame}


\section{Projects}
\label{sec:orgffcb831}

\begin{frame}[label={sec:orgfe3498d}]{Generative Prediction of Urban-scale Extreme Events: Project Idea I}
\begin{itemize}
\item 
\end{itemize}
\end{frame}

\begin{frame}[label={sec:org750d7a4}]{Generative Prediction of Urban-scale Extreme Events: Project Idea II}
\begin{itemize}
\item 
\end{itemize}
\end{frame}

\begin{frame}[label={sec:org51fbd96}]{Generative Prediction of Urban-scale Extreme Events: Results}
\begin{itemize}
\item 
\end{itemize}
\end{frame}

\begin{frame}[label={sec:org3d7f860}]{Generative Modeling of Kolmogorov Flow: Kolmogorov Method}
\begin{itemize}
\item 
\end{itemize}
\end{frame}

\begin{frame}[label={sec:org662a51f}]{Generative Modeling of Kolmogorov Flow: Data Assimilation Method}
\begin{itemize}
\item 
\end{itemize}
\end{frame}

\begin{frame}[label={sec:orgc5be97f}]{Generative Modeling of Kolmogorov Flow: Data Assimilation Results}
\begin{itemize}
\item 
\end{itemize}
\end{frame}

\begin{frame}[label={sec:orga1f7c43}]{Generative Modeling of Kolmogorov Flow: State Results}
\begin{itemize}
\item 
\end{itemize}
\end{frame}

\begin{frame}[label={sec:orgb5d6008}]{Generative Modeling for Windfarm Prediction: Reduced Order Modeling}
\begin{itemize}
\item 
\end{itemize}
\end{frame}

\begin{frame}[label={sec:org428b998}]{Generative Modeling for Windfarm Prediction: Neural Network Framework}
\begin{itemize}
\item 
\end{itemize}
\end{frame}

\begin{frame}[label={sec:orga570ddf}]{Generative Modeling for Windfarm Prediction: Results I}
\begin{itemize}
\item 
\end{itemize}
\end{frame}

\begin{frame}[label={sec:orga136011}]{Generative Modeling for Windfarm Prediction: Results II}
\begin{itemize}
\item 
\end{itemize}
\end{frame}


\section{Summary}
\label{sec:org6537d1d}
\begin{frame}[label={sec:org9542c36}]{Summary}
\begin{itemize}
\item 
\end{itemize}
\end{frame}


\appendix
\begin{frame}[label={sec:orgf0ab222}]{Thank you!}
\begin{center}
\alert{Thank you for your attention!}
\end{center}
\end{frame}

\begin{frame}[label={sec:org29c4560}]{References}
\printbibliography
\end{frame}
\end{document}
